% !TEX encoding = UTF-8 Unicode

%%%%%%%%%%%%%%%%%%%%%%% file plantilla.tex %%%%%%%%%%%%%%%%%%%%%%%%

\documentclass[runningheads,a4paper]{llncs}

\usepackage[utf8]{inputenc}
\usepackage[spanish, es-tabla]{babel}
\usepackage{emptypage}
\usepackage{graphicx}
\usepackage{amsmath} 
\usepackage{amssymb}
\usepackage{amsfonts}
\usepackage{amssymb}
\usepackage{captdef} 
\setcounter{tocdepth}{3}

\usepackage{booktabs}
\usepackage{microtype}



\usepackage{url}
\urldef{\mailsa}\path|{primer.autor, segundo.autor, tercer.autor}@mail.com|    
\newcommand{\keywords}[1]{\par\addvspace\baselineskip\noindent
\keywordname\enspace\ignorespaces#1}

\begin{document}

% primero se necesita el título
\title{Título de la contribución\thanks{Apoyado por la organización x}}

% Si el título del papel es demasiado largo, puede establecer un título de papel abreviado aquí
%\titlerunning{T\'{i}tulo de papel abreviado}


% el/los nombre(s) del(de los) autor(es) sigue(n) a continuación
\author{Primer Autor\inst{1} \and Segundo Autor\inst{2} \and
Tercer Autor\inst{2}\\}

% Los nombres se abrevian en la cabecera. Si hay más de dos autores se utiliza 'et al.'.
%\authorrunning{F. Author et al.}

% las afiliaciones se dan a continuación; no proporcione su dirección de correo electrónico a menos que acepte que se publicará
\institute{Universidad Politécnica de Madrid, Madrid, España \and
Universidad de Granada, Granada, España\\\mailsa}

\maketitle	% compilado el encabezado de la contribución

\begin{abstract}
El resumen o \textit{abstract} debe resumir el contenido del documento y debe contener al menos 70 y como máximo 150 palabras. Debe escribirse usando el entorno \emph{abstract}. Liste sus palabras clave dentro de esta sección de la siguiente manera:

\keywords{Primera palabra clave, Segunda palabra clave, ...}

\end{abstract}

\section{Introducci\'{o}n}
Le recomendamos usar \LaTeXe{} para la preparación de su contribución con el archivo de clase Springer correspondiente \verb+llncs.cls+. La fuente \LaTeX{} de este archivo de instrucciones se puede usar como plantilla. Esta se encuentra en el directorio \textbf{lncstemplate} y se titula \texttt{template.tex}.

Envíe el PDF final verificado de su documento a través del formulario habilitado para ello. Este formulario estará enlazado en la página web de las Jornadas de Software Libre de la Universidad de Granada \texttt{http://osl.ugr.es/JSLUGR/}. No es posible modificar un documento de ninguna manera, una vez que ha sido enviado. Todos los detalles, incluido el orden de los nombres de los autores, deben verificarse antes de enviar el documento.

\subsection{Verificado y env\'{i}o del archivo PDF}
Por favor, asegúrese de que en el formulario de envío de contribuciones rellena todos los campos correctamente. Es importante que revise la dirección de correo electrónico del autor del trabajo. Las propuestas aceptadas como póster, podrán traerlo para el desarrollo de la sesión pero no es necesario que se envíe en dicho formulario.

Se establecerá un plazo de envío de contribuciones que comienza con su aceptación en las Jornadas de Software Libre. La fecha de finalización del plazo de envío de trabajos se comunicará a los autores junto con el formulario de envío.

\subsection{Licencia empleada}
La contribución a las Jornadas de Software Libre de la Universidad de Granada debe estar protegida con una licencia \textit{Creative Commons}. Estas licencias están descritas en \texttt{https://creativecommons.org/}.

Para facilitar la elección de la licencia que se aplicará a su contribución, la página web oficial de licencias \textit{Creative Commons} cuenta con la siguiente herramienta de ayuda: \texttt{https://creativecommons.org/choose/}. Una vez escogida la licencia, asegúrese de que se trata de una licencia \textit{Creative Commons} totalmente libre. Esto se indica en el apartado \textbf{Licencia seleccionada}.

Para incorporar la licencia a su contribución debe hacer click sobre la licencia escogida. Esto le llevará a una página web que describe los derechos de dicha licencia. En la parte inferior aparece el link "utilice una licencia CC para su material". Si hace click en dicho enlace podrá acceder a los diferentes formatos de mencionar la licencia en función de su trabajo.En este caso, se trata de un documento escrito que puede no estar en la web por lo que debe seleccionar la opción de insertar texto en tu documento.

\section{Preparación de la contribución}
Su contribución a las Jornadas de Software Libre de la Universidad de Granada debe tener una extensión máxima de 6 páginas.

\subsection{Una muestra de subsección}
Tenga en cuenta que el primer párrafo de una sección o subsección no tiene sangría. El primer párrafo que sigue a una tabla, figura,
ecuación, etc., tampoco necesita una sangría.

Los párrafos posteriores, sin embargo, están sangrados.

Cuando utiliza el archivo de clase de documento \texttt{llncs.cls} de Springer (\texttt{https://www.springer.com/gp}), el texto se escribe automáticamente en la fuente \textit{Modern Computer Modern (CM)}.

\subsubsection{Muestra de encabezado de tercer nivel} 
Solo dos niveles de encabezados deben estar numerados. Los títulos de nivel inferior permanecen sin numerar; están formateados como encabezados internos.

\paragraph{Muestra de encabezado de cuarto nivel}
La contribución debe contener no más de cuatro niveles de encabezados. La tabla ~\ref{tab1} proporciona un resumen de todos los niveles de encabezado.

\subsection{Encabezados}
Los encabezados deben escribirse en mayúscula (es decir, sustantivos, verbos y todas las demás palabras, excepto los artículos, las preposiciones y las conjunciones, deben establecerse con un capital inicial) y deben, con la excepción del título, estar alineados a la izquierda. 

Las palabras unidas por un guión están sujetas a una regla especial. Si la primera palabra puede ser independiente, la segunda palabra debe estar en mayúscula.

\begin{table}
\caption{Las leyendas de las tablas deben colocarse encima de las tablas}\label{tab1}
\begin{tabular}{lll}
\toprule
Nivel de encabezado & Ejemplo & Tamaño de letra y estilo\\
\midrule
Título (centrado) &  {\Large\bfseries Notas de lectura} & 14 puntos, negrita\\
Título de primer nivel &  {\large\bfseries 1 Introducción} & 12 puntos, negrita\\
Título de segundo nivel & {\bfseries 1.1 Verificación y envío} & 10 puntos, negrita\\
Título de tercer nivel & {\bfseries Encabezado en negrita} El texto sigue & 10 puntos, negrita\\
Título de cuarto nivel & {\itshape Título de nivel más bajo} El texto sigue & 10 puntos, cursiva\\
\bottomrule
\end{tabular}
\end{table}

\subsection{Teoremas}
\begin{theorem}
Este es un teorema de muestra. El encabezado de entrada se establece en negrita, mientras que el siguiente texto aparece en cursiva. Las definiciones, lemas, proposiciones y corolarios se diseñan de la misma manera.
\end{theorem}

Los números de proposiciones y teoremas deben aparecer en orden consecutivo, comenzando con el teorema 1, y no, por ejemplo, con el teorema 11.

\begin{proof}
Las pruebas, los ejemplos y las observaciones tienen la palabra inicial en cursiva, mientras que el siguiente texto aparece en la fuente normal.
\end{proof}

\subsection{Citas}
Para citas de referencias, preferimos el uso de corchetes y números consecutivos. Las citas usando etiquetas o la convención autor/año también son aceptables. La bibliografía proporciona una lista de referencia de muestra con entradas para artículos de revistas~\cite{ref_article1}, un capítulo de LNCS~\cite{ref_lncs1}, un libro~\cite{ref_book1}, procedimientos sin editores~\cite{ref_proc1}, y una página de inicio~\cite{ref_url1}. Las citas múltiples son agrupadas~\cite{ref_article1,ref_lncs1,ref_book1}, \cite{ref_article1,ref_book1,ref_proc1,ref_url1}.

\subsection{Ecuaciones}
Las ecuaciones o fórmulas que se muestran están centradas y ubicadas en una línea separada (con una línea adicional o un espacio de línea media arriba y abajo). Las expresiones mostradas deben numerarse para referencia. Los números deben ser consecutivos dentro de cada sección o dentro de la contribución, con números entre paréntesis y en el margen derecho, que es el valor predeterminado si usa el entorno \emph{ecuación}, por ejemplo,
\begin{equation}
x + y = z
\end{equation}

\subsection{Pie de p\'{a}gina}

El número de superíndice utilizado para referirse a una nota al pie aparece en el texto directamente después de la palabra a discutir o, en relación con una frase o frase, siguiendo el signo de puntuación (coma, punto y coma o punto). Las notas al pie deben aparecer en la parte inferior del área de texto normal, con una línea de 2 cm aproximadamente colocada inmediatamente encima de ellas. \footnote{número de la nota al pie se establece a la izquierda y el texto sigue con el espaciado de palabras habitual.}

\subsection{Figuras}
Intente evitar las imágenes rasterizadas para diagramas y esquemas de líneas de arte. Siempre que sea posible, use gráficos vectoriales en su lugar (vea la Fig.~\ref{eijkel2}).

Verifique que las líneas en los dibujos de líneas no estén interrumpidas y tengan un ancho constante. Las cuadrículas y los detalles dentro de las figuras deben ser claramente legibles y no pueden escribirse uno encima del otro. Los dibujos lineales deben tener una resolución de al menos 800 ppp (preferiblemente 1200 ppp). Las letras en las figuras deben tener una altura de 2 mm (tipo de 10 puntos). Las figuras deben estar numeradas y deben tener un título que siempre debe colocarse \emph{debajo} de las figuras, en contraste con el título que pertenece a una tabla, que siempre debe aparecer \emph{encima} de la tabla; esto simplemente se logra como una cuestión de secuencia en su fuente.

\begin{figure}
\centering
\includegraphics[height=6.2cm]{eijkel2}
\caption{La leyenda de una figura siempre se coloca debajo de la ilustración. Tenga en cuenta que los subtítulos cortos están centrados, mientras que los largos son justificado por el macro paquete de forma automática.}
\label{fig:ejemplo}
\end{figure}

Por favor, centre las figuras usando la declaración \verb+\centering+. Los subtítulos cortos están centrados por defecto entre los márgenes y el tipo en tipo de 9 puntos. La distancia entre el texto y la figura está preestablecida en aproximadamente 8 mm, la distancia entre la figura y el título es de aproximadamente 6 mm.

\subsection{C\'{o}digo de un programa}

Las listas de programas o los comandos de programa en el texto normalmente se establecen en letra de máquina de escribir, por ejemplo, CMTT10 o Courier.

\medskip
\noindent
\textit{Ejemplo de un programa de computadora}
\begin{verbatim}
program Inflation (Output)
  {Assuming annual inflation rates of 7%, 8%, and 10%,...
   years};
   const
     MaxYears = 10;
   var
     Year: 0..MaxYears;
     Factor1, Factor2, Factor3: Real;
   begin
     Year := 0;
     Factor1 := 1.0; Factor2 := 1.0; Factor3 := 1.0;
     WriteLn('Year  7% 8% 10%'); WriteLn;
     repeat
       Year := Year + 1;
       Factor1 := Factor1 * 1.07;
       Factor2 := Factor2 * 1.08;
       Factor3 := Factor3 * 1.10;
       WriteLn(Year:5,Factor1:7:3,Factor2:7:3,Factor3:7:3)
     until Year = MaxYears
end.
\end{verbatim}

\noindent {\small (Ejemplo de Jensen K., Wirth N. (1991) Manual de usuario de Pascal e informe. Springer, New York)}

% ---- Bibliografía ----
% Los usuarios de BibTeX deben especificar el estilo de bibliografía 'splncs04'.
% Las referencias serán ordenadas y formateadas en el estilo correcto.

\bibliographystyle{splncs04}

\begin{thebibliography}{4}
\bibitem{ref_article1}
Autor, F.: Título del artículo. Revista \textbf{2}(5), 99--110 (2016)

\bibitem{ref_lncs1}
Autor, F., Autor, S.: Título de un documento de procedimiento. In: Editor, F., Editor, S. (eds.) CONFERENCE 2016, LNCS, vol. 9999, pp. 1--13. Springer, Heidelberg (2016).

\bibitem{ref_book1}
Autor, F., Autor, S., Autor, T.: Título del libro. 2nd edn. Editor, Ubicación (1999)

\bibitem{ref_proc1}
Autor, A.-B.: Título de la Contribución. In: 9th International Proceedings on Proceedings, pp. 1--2. Editor, Ubicación (2010)

\bibitem{ref_url1}
Página web LNCS, \url{http://www.springer.com/lncs}. Último acceso 4 Oct 2017
\end{thebibliography}

\section*{Anexo 1}

\end{document}

